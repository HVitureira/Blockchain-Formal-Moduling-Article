%%
%% This is file `article.tex',
%%
%% Commands for TeXCount
%TC:macro \cite [option:text,text]
%TC:macro \citep [option:text,text]
%TC:macro \citet [option:text,text]
%TC:envir table 0 1
%TC:envir table* 0 1
%TC:envir tabular [ignore] word
%TC:envir displaymath 0 word
%TC:envir math 0 word
%TC:envir comment 0 0
%%
%%
%% The first command in your LaTeX source must be the \documentclass command.
\documentclass[sigconf, nonacm]{acmart}

%%
%% \BibTeX command to typeset BibTeX logo in the docs
\AtBeginDocument{%
  \providecommand\BibTeX{{%
    \normalfont B\kern-0.5em{\scshape i\kern-0.25em b}\kern-0.8em\TeX}}}

%% Rights management information.  This information is sent to you
%% when you complete the rights form.  These commands have SAMPLE
%% values in them; it is your responsibility as an author to replace
%% the commands and values with those provided to you when you
%% complete the rights form.
\setcopyright{rightsretained}
\copyrightyear{2021}
%\settopmatter{printccs=true, printacmref=true}

\usepackage{listings}

%%
%% end of the preamble, start of the body of the document source.
\begin{document}

\title{Literature review of Formal Methods applied to Blockchain technology}

\author{Henoch Vitureira}
\affiliation{%
  \institution{Polytechnic Institute of Setubal}
  \streetaddress{Campus do IPS - Estefanilha, 2910-761}
  \city{Setúbal}
  \country{Portugal}}
\email{2016041081@estudantes.ips.pt}


%%
%% By default, the full list of authors will be used in the page
%% headers. Often, this list is too long, and will overlap
%% other information printed in the page headers. This command allows
%% the author to define a more concise list
%% of authors' names for this purpose.
\renewcommand{\shortauthors}{Henoch Vitureira}

%%
%% The abstract is a short summary of the work to be presented in the
%% article.
\begin{abstract}
  Lorem ipsum dolor sit amet, consectetur adipiscing elit. 
  Vestibulum enim nibh, aliquet sit amet ex suscipit, efficitur fringilla enim. 
  Vestibulum elementum fringilla sodales. Cras malesuada sollicitudin consectetur.
  Cras elit nibh, tempus elementum dui vitae, sagittis porttitor elit. In sollicitudin purus neque, eu tristique felis varius in. Morbi sed commodo eros.
  Pellentesque ornare sed felis et bibendum.
\end{abstract}

%%
%% The code below is generated by the tool at http://dl.acm.org/ccs.cfm.
%% Please copy and paste the code instead of the example below.
%%
\begin{CCSXML}
  <ccs2012>
  <concept>
  <concept_id>10011007.10011074.10011099.10011105</concept_id>
  <concept_desc>Software and its engineering~Process validation</concept_desc>
  <concept_significance>500</concept_significance>
  </concept>
  <concept>
  <concept_id>10002978.10002979</concept_id>
  <concept_desc>Security and privacy~Cryptography</concept_desc>
  <concept_significance>500</concept_significance>
  </concept>
  </ccs2012>
\end{CCSXML}

\ccsdesc[500]{Software and its engineering~Process validation}
\ccsdesc[500]{Security and privacy~Cryptography}

%%
%% Keywords. The author(s) should pick words that accurately describe
%% the work being presented. Separate the keywords with commas.
\keywords{Software Quality, Formal Methods, Blockchain, Smart contracts}


%%
%% This command processes the author and affiliation and title
%% information and builds the first part of the formatted document.
\maketitle

\section{Introduction}
The rising complexity of features offered by software gives us the ability to use it for the most integrity and reliability dependent needs of our lives. 
Information systems that rely on these base principles can be financial and health related services, and in most recent times, blockchain based applications. 
Blockchain, being a decentralized and distributed network protocol that can be used as the core of a monetary system that performs peer-to-peer transactions, 
it needs the assurance of coordination and consensus of its economy’s state~\cite{Duan2018}. 
Knowing that the protocol of a given blockchain project is what dictates how transactions and their coordination is maintained, the assurance of security and safety is a given. 
Formal methods can be introduced in order to provide unequivocal evidence that a given blockchain system and its consensus algorithm are secure and in according to expected software quality.

Formal methods are a rigorous description of a system or process using mathematics that aims to provide evidence of its reliability and robustness, 
according the specification in question~\cite{Regan2014}.

//TODO FINISH

\section{Formal Modeling Application}
Blockchain systems can be divided into five main layers~\cite{Duan2018}: Data layer, Consensus layer, Smart contract layer and Application layer. 
As seen on Figure~\ref{fig:blockchain-layers}.

  \begin{figure}[h]
    \centering
    \includegraphics[width=\linewidth]{blockchain-layers}
    \caption{Blockchain infrastructure, from \cite{Duan2018}}
    \Description{Ilustration of the layers that describe a blockchain system}
    \label{fig:blockchain-layers}
  \end{figure}

The data layer is responsible for block generation, the construction of the blockchain and storage~\cite{Duan2018}. 
The network layer connects the data and filters packets, and also manages the nodes. 
The rules of how the blockchain protocol must work are defined on the consensus layer. 
This includes agreement between nodes, fault tolerance and data consistency. 
Smart contracts have their execution lifecycle on the smart contact layer.
The application layer is the upper level layer of used applications such as games or decentralized financial services. 

It is possible to use languages to define the standards of a blockchain system with mathematical rigor.
One of them is SDL (Specification and Description Language), an Object-Oriented formal language defined by the International Telecommunications Unit.
It is capable of mathematically describing the structure, behavior, and data of real-time and distributed communicating systems. 
It is also used by IEEE for standard definition~\cite{Duan2018}.

With SDL, we can describe a structure at the levels of system, block or process. A system represents the object to describe. 
It is responsible for the communication with the outer environment through the appropriate channels.
A system can contain one or more blocks and blocks contain one or more processes.
The majority of the functionality of an SDL system is described on the process level.
Comparing to conventional programming languages, modules are represented by blocks in SDL, and funcionalities are represented by processes.
In~\cite{Duan2018}, the authors propose a hierarchical model focused on the consensus layer.

  \subsection{Data Structure}
  In~\cite{Duan2018} the modelled blockchain system used an Improved Byzantine Fault Tolerant Algorithm as the consensus algorithm.
  To define the data structure of a block in SDL, we need to represent them as a pair with a header BT and body BH, where:

  \begin{displaymath}
    B~\equiv(B_H,B_T) \quad \textrm{with} \quad B_H~\equiv(\mathbb{H}_p,\mathbb{H}_a,\mathbb{H}_t,\mathbb{H}_m,\mathbb{H}_h) 
  \end{displaymath}
  \begin{displaymath}
    \quad \textrm{and} \quad B_T~\equiv(T_1,T_2,...,T_{\mathbb{H}_l}) 
  \end{displaymath}
  \linebreak[1]

  In these expressions, $\mathbb{H}_p$ represents the hash value of the previous block, 
  $\mathbb{H}_p$ the hash value of the current block, $\mathbb{H}_t$ it's timestamp, 
  $\mathbb{H}_m$ the merkle tree root (the hash of the transactions of a block), and $\mathbb{H}_h$ the current block's height.

  The the block' structure definition can be seen on Listing~\ref{lst:block-code}.
  
  \begin{lstlisting}[frame=single,label={lst:block-code},
    caption={Definition of a block's structure},captionpos=b]
    newtype blockchain struct
      prehash integer;
      hash integer;
      length integer;/*the length of blocks*/
      merkleroot integer;
      ti Time;/*Timestamp*/
      translist list;
    endnewtype;

    newtype list
     array(maxit, integer)
    endnewtype;
 
    syntype maxit =Integer constants 0:25
    endsyntype;
  \end{lstlisting}
  
  A blockchain can be regarded as a state machine~\cite{Duan2018} that contains a starting state, a non-empty state, the input transaction set,
  a state transition function, and an acceptance state set. Which can be formally defined as:
    \begin{displaymath}
      S~\equiv(Q,\Sigma,\delta,s,F)
    \end{displaymath}

  Where:

  \begin{enumerate}
    \item[$\blacksquare$] Q is the non-empty state set, which represents all the states.
    \item[$\blacksquare$] $\Sigma$ is the set of the newly generated and consensus blocks. 
    \item[$\blacksquare$] $\delta$ represents the state transition function,
      \begin{math}
        \delta: Q \times \Sigma \rightarrow Q
      \end{math}.
    \item[$\blacksquare$] s would be the starting state, which is the state of the system when it initializes.
      \begin{math}
        \delta: s \in Q
      \end{math}.
    \item[$\blacksquare$] F representes the acceptance state set.
      \begin{math}
        F \subseteq Q
      \end{math}.
  \end{enumerate}
  \  \\ %fake break line
  The state of the model at the first block generation is the starting state.
  As transactions are made, the model a leader node to generate the block, which is transmitted to the other nodes on the blockchain network,
  triggering its validation for consensus. 
  A consensus block stores the hash of the previous one and adds it to the end of the blockchain and completes the transfer of the blockchain state.
  
  \subsubsection{Diagrams}
  With a modelling language like SDL, we can use diagrams to define blockchain structural entities, like the core system of the consensus layer
  and blocks. Process flowcharts can model the flow of information and input validation, making a process like
  node validations of blocks and its subprocess of synchronization and voting can clearly illustrated.

\section{Consensus Rule}
Blockchain system may choose to use consensus rules taken from popular protocols such as Bitcoin,
but an alternative rule can be implemented, and also be formally modelled~\cite{Kawahara2020}.

On a consensus rule, fault tolerances requirements are of utmost importance, including the collusion of nodes to alter a ledger.
In~\cite{Kawahara2020}, the author discusses a way to formally describe these tolerances with a scalable verification.

Endorsement policies, the rule that define the criteria of agreement os transaction results between nodes, can be described
as a threshold function as follows:

\begin{displaymath}
  T(m,\textrm{node}_1,... ,\textrm{node}_n), m \in \mathbb{N}
\end{displaymath}
\linebreak[1]

Where $T(2,\textrm{node}_1,\textrm{node}_2,\textrm{node}_3)$ would be an endorsement policy
that means "among the three nodes in the argument, at least two nodes must return the same execution result".
With this, we can mathematically model faulty state verification.

\subsection{Faulty states}
Let $\mathbb{B}$ and $\mathbb{N}$ be sets of Boolean and natural numbers, respectively. The state of the \textit{i}-th node can be defined as:
$s_i = (s_i.e,s_i.v)$, where for each transaction execution, $s_i.e \in \mathbb{B}$ is the  existence of the result,
and $s_i.v \in \mathbb{B}$ is the value of the result (correct or wrong).

Using the threshold function previously shown, $T(m,s_1,... ,s_n) = (T.e,T.v)$. With this, let

\begin{displaymath}
  H_ = \left(\sum_{i}^{n} I(s_i.e \wedge (s_i.v = b)) \geq m \right)
\end{displaymath}

Where:

\begin{enumerate}
  \item[$\blacksquare$] \textit{m} is a treshold
  \item[$\blacksquare$] $b \in \mathbb{B}$
  \item[$\blacksquare$] $I : \mathbb{B} \mapsto \{0,1\}$  is the indicator function
\end{enumerate}
\  \\ %fake break line

With the expression above, the author of~\cite{Kawahara2020} infers:

\begin{displaymath}
  T.e = H_{true} \vee H_{false},  T.v = \neg{H_{false}}
\end{displaymath}

The  endorsement policy $EP = (EP.e,EP.v)$ can be formed with \textit{T}. When $EP.e = \textrm{true}$, the policy
is considered accordingly followed and the involved ledger is updated with the value of $EP.v$.

\section{Template Overview}
As noted in the introduction, the ``\verb|acmart|'' document class can
be used to prepare many different kinds of documentation --- a
double-blind initial submission of a full-length technical paper, a
two-page SIGGRAPH Emerging Technologies abstract, a ``camera-ready''
journal article, a SIGCHI Extended Abstract, and more --- all by
selecting the appropriate {\itshape template style} and {\itshape
    template parameters}.

This document will explain the major features of the document
class. For further information, the {\itshape \LaTeX\ User's Guide} is
available from
\url{https://www.acm.org/publications/proceedings-template}.

\subsection{Template Styles}

The primary parameter given to the ``\verb|acmart|'' document class is
the {\itshape template style} which corresponds to the kind of publication
or SIG publishing the work. This parameter is enclosed in square
brackets and is a part of the {\verb|documentclass|} command:
\begin{verbatim}
  \documentclass[STYLE]{acmart}
\end{verbatim}

Journals use one of three template styles. All but three ACM journals
use the {\verb|acmsmall|} template style:
\begin{itemize}
  \item {\verb|acmsmall|}: The default journal template style.
  \item {\verb|acmlarge|}: Used by JOCCH and TAP.
  \item {\verb|acmtog|}: Used by TOG.
\end{itemize}

The majority of conference proceedings documentation will use the {\verb|acmconf|} template style.
\begin{itemize}
  \item {\verb|acmconf|}: The default proceedings template style.
        \item{\verb|sigchi|}: Used for SIGCHI conference articles.
        \item{\verb|sigchi-a|}: Used for SIGCHI ``Extended Abstract'' articles.
        \item{\verb|sigplan|}: Used for SIGPLAN conference articles.
\end{itemize}

\subsection{Template Parameters}

In addition to specifying the {\itshape template style} to be used in
formatting your work, there are a number of {\itshape template parameters}
which modify some part of the applied template style. A complete list
of these parameters can be found in the {\itshape \LaTeX\ User's Guide.}

Frequently-used parameters, or combinations of parameters, include:
\begin{itemize}
  \item {\verb|anonymous,review|}: Suitable for a ``double-blind''
        conference submission. Anonymizes the work and includes line
        numbers. Use with the \verb|\acmSubmissionID| command to print the
        submission's unique ID on each page of the work.
        \item{\verb|authorversion|}: Produces a version of the work suitable
        for posting by the author.
        \item{\verb|screen|}: Produces colored hyperlinks.
\end{itemize}

This document uses the following string as the first command in the
source file:
\begin{verbatim}
\documentclass[sigconf]{acmart}
\end{verbatim}

\section{Modifications}

Modifying the template --- including but not limited to: adjusting
margins, typeface sizes, line spacing, paragraph and list definitions,
and the use of the \verb|\vspace| command to manually adjust the
vertical spacing between elements of your work --- is not allowed.

  {\bfseries Your document will be returned to you for revision if
    modifications are discovered.}

\section{Typefaces}

The ``\verb|acmart|'' document class requires the use of the
``Libertine'' typeface family. Your \TeX\ installation should include
this set of packages. Please do not substitute other typefaces. The
``\verb|lmodern|'' and ``\verb|ltimes|'' packages should not be used,
as they will override the built-in typeface families.

\section{Title Information}

The title of your work should use capital letters appropriately -
\url{https://capitalizemytitle.com/} has useful rules for
capitalization. Use the {\verb|title|} command to define the title of
your work. If your work has a subtitle, define it with the
  {\verb|subtitle|} command.  Do not insert line breaks in your title.

If your title is lengthy, you must define a short version to be used
in the page headers, to prevent overlapping text. The \verb|title|
command has a ``short title'' parameter:
\begin{verbatim}
  \title[short title]{full title}
\end{verbatim}

\section{Authors and Affiliations}

Each author must be defined separately for accurate metadata
identification. Multiple authors may share one affiliation. Authors'
names should not be abbreviated; use full first names wherever
possible. Include authors' e-mail addresses whenever possible.

Grouping authors' names or e-mail addresses, or providing an ``e-mail
alias,'' as shown below, is not acceptable:
\begin{verbatim}
  \author{Brooke Aster, David Mehldau}
  \email{dave,judy,steve@university.edu}
  \email{firstname.lastname@phillips.org}
\end{verbatim}

The \verb|authornote| and \verb|authornotemark| commands allow a note
to apply to multiple authors --- for example, if the first two authors
of an article contributed equally to the work.

If your author list is lengthy, you must define a shortened version of
the list of authors to be used in the page headers, to prevent
overlapping text. The following command should be placed just after
the last \verb|\author{}| definition:
\begin{verbatim}
  \renewcommand{\shortauthors}{McCartney, et al.}
\end{verbatim}
Omitting this command will force the use of a concatenated list of all
of the authors' names, which may result in overlapping text in the
page headers.

The article template's documentation, available at
\url{https://www.acm.org/publications/proceedings-template}, has a
complete explanation of these commands and tips for their effective
use.

Note that authors' addresses are mandatory for journal articles.

\section{Rights Information}

Authors of any work published by ACM will need to complete a rights
form. Depending on the kind of work, and the rights management choice
made by the author, this may be copyright transfer, permission,
license, or an OA (open access) agreement.

Regardless of the rights management choice, the author will receive a
copy of the completed rights form once it has been submitted. This
form contains \LaTeX\ commands that must be copied into the source
document. When the document source is compiled, these commands and
their parameters add formatted text to several areas of the final
document:
\begin{itemize}
  \item the ``ACM Reference Format'' text on the first page.
  \item the ``rights management'' text on the first page.
  \item the conference information in the page header(s).
\end{itemize}

Rights information is unique to the work; if you are preparing several
works for an event, make sure to use the correct set of commands with
each of the works.

The ACM Reference Format text is required for all articles over one
page in length, and is optional for one-page articles (abstracts).

\section{CCS Concepts and User-Defined Keywords}

Two elements of the ``acmart'' document class provide powerful
taxonomic tools for you to help readers find your work in an online
search.

The ACM Computing Classification System ---
\url{https://www.acm.org/publications/class-2012} --- is a set of
classifiers and concepts that describe the computing
discipline. Authors can select entries from this classification
system, via \url{https://dl.acm.org/ccs/ccs.cfm}, and generate the
commands to be included in the \LaTeX\ source.

User-defined keywords are a comma-separated list of words and phrases
of the authors' choosing, providing a more flexible way of describing
the research being presented.

CCS concepts and user-defined keywords are required for for all
articles over two pages in length, and are optional for one- and
two-page articles (or abstracts).

\section{Sectioning Commands}

Your work should use standard \LaTeX\ sectioning commands:
\verb|section|, \verb|subsection|, \verb|subsubsection|, and
\verb|paragraph|. They should be numbered; do not remove the numbering
from the commands.

Simulating a sectioning command by setting the first word or words of
a paragraph in boldface or italicized text is {\bfseries not allowed.}

\section{Tables}

The ``\verb|acmart|'' document class includes the ``\verb|booktabs|''
package --- \url{https://ctan.org/pkg/booktabs} --- for preparing
high-quality tables.

Table captions are placed {\itshape above} the table.

Because tables cannot be split across pages, the best placement for
them is typically the top of the page nearest their initial cite.  To
ensure this proper ``floating'' placement of tables, use the
environment \textbf{table} to enclose the table's contents and the
table caption.  The contents of the table itself must go in the
\textbf{tabular} environment, to be aligned properly in rows and
columns, with the desired horizontal and vertical rules.  Again,
detailed instructions on \textbf{tabular} material are found in the
\textit{\LaTeX\ User's Guide}.

Immediately following this sentence is the point at which
Table~\ref{tab:freq} is included in the input file; compare the
placement of the table here with the table in the printed output of
this document.

\begin{table}
  \caption{Frequency of Special Characters}
  \label{tab:freq}
  \begin{tabular}{ccl}
    \toprule
    Non-English or Math & Frequency   & Comments          \\
    \midrule
    \O                  & 1 in 1,000  & For Swedish names \\
    $\pi$               & 1 in 5      & Common in math    \\
    \$                  & 4 in 5      & Used in business  \\
    $\Psi^2_1$          & 1 in 40,000 & Unexplained usage \\
    \bottomrule
  \end{tabular}
\end{table}

To set a wider table, which takes up the whole width of the page's
live area, use the environment \textbf{table*} to enclose the table's
contents and the table caption.  As with a single-column table, this
wide table will ``float'' to a location deemed more
desirable. Immediately following this sentence is the point at which
Table~\ref{tab:commands} is included in the input file; again, it is
instructive to compare the placement of the table here with the table
in the printed output of this document.

\begin{table*}
  \caption{Some Typical Commands}
  \label{tab:commands}
  \begin{tabular}{ccl}
    \toprule
    Command                    & A Number & Comments         \\
    \midrule
    \texttt{{\char'134}author} & 100      & Author           \\
    \texttt{{\char'134}table}  & 300      & For tables       \\
    \texttt{{\char'134}table*} & 400      & For wider tables \\
    \bottomrule
  \end{tabular}
\end{table*}

Always use midrule to separate table header rows from data rows, and
use it only for this purpose. This enables assistive technologies to
recognise table headers and support their users in navigating tables
more easily.

\section{Math Equations}
You may want to display math equations in three distinct styles:
inline, numbered or non-numbered display.  Each of the three are
discussed in the next sections.

\subsection{Inline (In-text) Equations}
A formula that appears in the running text is called an inline or
in-text formula.  It is produced by the \textbf{math} environment,
which can be invoked with the usual
\texttt{{\char'134}begin\,\ldots{\char'134}end} construction or with
the short form \texttt{\$\,\ldots\$}. You can use any of the symbols
and structures, from $\alpha$ to $\omega$, available in
\LaTeX~\cite{Lamport:LaTeX}; this section will simply show a few
examples of in-text equations in context. Notice how this equation:
\begin{math}
  \lim_{n\rightarrow \infty}x=0
\end{math},
set here in in-line math style, looks slightly different when
set in display style.  (See next section).

\subsection{Display Equations}
A numbered display equation---one set off by vertical space from the
text and centered horizontally---is produced by the \textbf{equation}
environment. An unnumbered display equation is produced by the
\textbf{displaymath} environment.

Again, in either environment, you can use any of the symbols and
structures available in \LaTeX\@; this section will just give a couple
of examples of display equations in context.  First, consider the
equation, shown as an inline equation above:
\begin{equation}
  \lim_{n\rightarrow \infty}x=0
\end{equation}
Notice how it is formatted somewhat differently in
the \textbf{displaymath}
environment.  Now, we'll enter an unnumbered equation:
\begin{displaymath}
  \sum_{i=0}^{\infty} x + 1
\end{displaymath}
and follow it with another numbered equation:
\begin{equation}
  \sum_{i=0}^{\infty}x_i=\int_{0}^{\pi+2} f
\end{equation}
just to demonstrate \LaTeX's able handling of numbering.

\section{Figures}

The ``\verb|figure|'' environment should be used for figures. One or
more images can be placed within a figure. If your figure contains
third-party material, you must clearly identify it as such, as shown
in the example below.
\begin{figure}[h]
  \centering
  \includegraphics[width=\linewidth]{sample-franklin}
  \caption{1907 Franklin Model D roadster. Photograph by Harris \&
    Ewing, Inc. [Public domain], via Wikimedia
    Commons. (\url{https://goo.gl/VLCRBB}).}
  \Description{A woman and a girl in white dresses sit in an open car.}
\end{figure}

Your figures should contain a caption which describes the figure to
the reader.

Figure captions are placed {\itshape below} the figure.

Every figure should also have a figure description unless it is purely
decorative. These descriptions convey what’s in the image to someone
who cannot see it. They are also used by search engine crawlers for
indexing images, and when images cannot be loaded.

A figure description must be unformatted plain text less than 2000
characters long (including spaces).  {\bfseries Figure descriptions
    should not repeat the figure caption – their purpose is to capture
    important information that is not already provided in the caption or
    the main text of the paper.} For figures that convey important and
complex new information, a short text description may not be
adequate. More complex alternative descriptions can be placed in an
appendix and referenced in a short figure description. For example,
provide a data table capturing the information in a bar chart, or a
structured list representing a graph.  For additional information
regarding how best to write figure descriptions and why doing this is
so important, please see
\url{https://www.acm.org/publications/taps/describing-figures/}.

\subsection{The ``Teaser Figure''}

A ``teaser figure'' is an image, or set of images in one figure, that
are placed after all author and affiliation information, and before
the body of the article, spanning the page. If you wish to have such a
figure in your article, place the command immediately before the
\verb|\maketitle| command:
\begin{verbatim}
  \begin{teaserfigure}
    \includegraphics[width=\textwidth]{sampleteaser}
    \caption{figure caption}
    \Description{figure description}
  \end{teaserfigure}
\end{verbatim}

\section{Citations and Bibliographies}

The use of \BibTeX\ for the preparation and formatting of one's
references is strongly recommended. Authors' names should be complete
--- use full first names (``Donald E. Knuth'') not initials
(``D. E. Knuth'') --- and the salient identifying features of a
reference should be included: title, year, volume, number, pages,
article DOI, etc.

The bibliography is included in your source document with these two
commands, placed just before the \verb|\end{document}| command:
\begin{verbatim}
  \bibliographystyle{ACM-Reference-Format}
  \bibliography{bibfile}
\end{verbatim}
where ``\verb|bibfile|'' is the name, without the ``\verb|.bib|''
suffix, of the \BibTeX\ file.

Citations and references are numbered by default. A small number of
ACM publications have citations and references formatted in the
``author year'' style; for these exceptions, please include this
command in the {\bfseries preamble} (before the command
``\verb|\begin{document}|'') of your \LaTeX\ source:
\begin{verbatim}
  \citestyle{acmauthoryear}
\end{verbatim}

Some examples.  A paginated journal article \cite{Afzaal2021}, an
enumerated journal article \cite{Murray2019}, a reference to an entire
issue \cite{Matsuo2017}, a monograph (whole book) \cite{Duan2018}, a
monograph/whole book in a series (see 2a in spec. document)
\cite{Kawahara2020}, a divisible-book such as an anthology or compilation
\cite{Liu2019} followed by the same example, however we only output
the series if the volume number is given.


%%
%% The next two lines define the bibliography style to be used, and
%% the bibliography file.
\bibliographystyle{ACM-Reference-Format}
\bibliography{article-base}

%%
%% If your work has an appendix, this is the place to put it.
\appendix

\section{Research Methods}

\subsection{Part One}

Lorem ipsum dolor sit amet, consectetur adipiscing elit. Morbi
malesuada, quam in pulvinar varius, metus nunc fermentum urna, id
sollicitudin purus odio sit amet enim. Aliquam ullamcorper eu ipsum
vel mollis. Curabitur quis dictum nisl. Phasellus vel semper risus, et
lacinia dolor. Integer ultricies commodo sem nec semper.

\subsection{Part Two}

Etiam commodo feugiat nisl pulvinar pellentesque. Etiam auctor sodales
ligula, non varius nibh pulvinar semper. Suspendisse nec lectus non
ipsum convallis congue hendrerit vitae sapien. Donec at laoreet
eros. Vivamus non purus placerat, scelerisque diam eu, cursus
ante. Etiam aliquam tortor auctor efficitur mattis.

\section{Online Resources}

Nam id fermentum dui. Suspendisse sagittis tortor a nulla mollis, in
pulvinar ex pretium. Sed interdum orci quis metus euismod, et sagittis
enim maximus. Vestibulum gravida massa ut felis suscipit
congue. Quisque mattis elit a risus ultrices commodo venenatis eget
dui. Etiam sagittis eleifend elementum.

Nam interdum magna at lectus dignissim, ac dignissim lorem
rhoncus. Maecenas eu arcu ac neque placerat aliquam. Nunc pulvinar
massa et mattis lacinia.

\end{document}
\endinput
%%
%% End of file `article.tex'.
